\documentclass[12pt,letterpaper,oneside,notitlepage,onecolumn]{article}%
\usepackage[utf8]{inputenc}%
\usepackage{textcomp}%
\usepackage{lastpage}%
\usepackage[margin=0.69in]{geometry}%
%
\title{Vector{-}Rotation to Euler Angles}%
\author{Buck Baskin}%
\date{\today}%
\usepackage[english]{babel}%
\usepackage[margin=0.69in]{geometry}%
\usepackage{parskip}%
\usepackage{listings}%
\usepackage{color}%
\usepackage{verbatim}%
\usepackage{soul}%
\usepackage{amsmath}%
\usepackage{amssymb}%
\usepackage{amsthm}%
\usepackage{gensymb}%
\usepackage{graphicx}%
\definecolor{dkgreen}{rgb}{0, 0.6, 0}%
\definecolor{gray}{rgb}{0.5, 0.5, 0.5}%
\definecolor{mauve}{rgb}{0.58, 0, 0.82}%
\lstset{frame=tb,
  language=Matlab,
  aboveskip=3mm,
  belowskip=3mm,
  showstringspaces=false,
  columns=flexible,
  basicstyle={\small\ttfamily},
  numbers=none,
  numberstyle=\tiny\color{gray},
  keywordstyle=\color{blue},
  commentstyle=\color{dkgreen},
  stringstyle=\color{mauve},
  breaklines=true,
  breakatwhitespace=true,
  tabsize=3
}%
\DeclareMathOperator*{\argmax}{arg\,max}%
\DeclareMathOperator*{\argmin}{arg\,min}%
\newcommand{\subsubsubsection}{\paragraph}%
\newcommand{\bbs}[1]{\section{#1}}%
\newcommand{\bbbs}[1]{\subsection{#1}}%
\newcommand{\bbbbs}[1]{\subsubsection{#1}}%
\newcommand{\bbbbbs}[1]{\subsubsubsection{#1}}%
\newcommand{\norm}[1]{\left\lVert#1\right\rVert}%
%
\begin{document}%
\normalsize%
\maketitle%
Some calculations were verified using Python.%
\section{Answer}%
$R_{BA} = R_{y}(\alpha) R_{x}(\beta) R_{y}(\gamma)$%
\subsection{Solving for $\alpha, \beta, \gamma$}%
\subsubsection{$\beta$}%
Using the atan method, one can solve for $\beta$ using the middle column of the matrix. The middle value, $r_{22} = cos(\beta)$. Using $r_{12}^{2} + r_{32}^{2} = sin^{2}(\beta)(sin^{2}(\gamma) + cos^{2}(\gamma)) = sin^{2}(\beta)$, $sin(\beta) = \sqrt{r_{12}^{2} + r_{32}^{2}}$. Therefore, $\beta = atan2(\sqrt{r_{12}^{2} + r_{32}^{2}}, r_{22})$.

%
\subsubsection{$\gamma$}%
Looking at two elements in the middle row of the matrix, one can solve for $\gamma$ using the atan method. $r_{21} = sin(\beta) sin(\gamma)$, so $sin(\gamma) = r_{21} / sin(\beta)$. $r_{23} = -sin(\beta) cos(\gamma)$, so $cos(\gamma) = -r_{23} / sin(\beta)$. Therefore, $\gamma = atan2(r_{21} / sin(\beta), -r_{23} / sin(\beta))$.

%
\subsubsection{$\alpha$}%
$r_{12} = sin(\alpha) sin(\beta)$, $sin(\alpha) = r_{12} / sin(\beta)$\\$r_{32} = cos(\alpha) sin(\beta)$, $cos(\alpha) = r_{32} / sin(\beta)$\\$\alpha = atan2(r_{12} / sin(\beta), r_{32} / sin(\beta))$

%
\subsection{Conclusion}%
Substituting the correct values from $R_{BA}$ calculated by Rodriguez Formula into the spaces allocated by $r_{ij}$ that were solved into an $atan2$ formula using $R_{BA}$ calculated by the rotation matrices gives the solution for the equivalent rotations in radians (rounded to 3 decimal places).%
\newline%
$\alpha = 1.833, \beta = 1.231, \gamma = -2.880$

%
\end{document}